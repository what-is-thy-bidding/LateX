\documentclass{article}
\usepackage{blindtext}
\usepackage[T1]{fontenc}
\usepackage[utf8]{inputenc}
\usepackage{indentfirst}
\usepackage{changepage}
\usepackage{amsmath}
\usepackage[utf8]{inputenc}
\usepackage{longtable}
\usepackage{graphicx}
\usepackage{verbatim}
\usepackage{times}
\usepackage{bm}
\usepackage{amsmath,amssymb}
\newenvironment{subs}
  {\adjustwidth{3em}{0pt}}
  {\endadjustwidth}

\title{Assignment 2}
\author{Akshat Bisht 40053762}
\date{\today}
 
\usepackage{tikz}

\usetikzlibrary{shapes}



\begin{document}

\title{Chapter1 (Recommended Exercises)}
\author{Akshat Bisht}
\date{\today}

\maketitle

%----------------------------------------------------------------------------------------------------------------------------------------
\section*{ 3.Randomized Ski Rental Problem } 
%\begin{subs}
	\begin{quote}
		Given: \\
		P(buy ski on the 1st day) = $\frac{b-1}{b}$ ;  P(rent and never buy)=$\frac{1}{b} $\\
		Cost of buying=b and Cost of renting=1\\
		Weather spoils on day \textit{k}.
		
		1. Expected cost of algorithm in terms of \textit{b} and \textit{k}?\\
		\begin{quote}
			
			\begin{center}
				\begin{tabular}{ | c | c| }
					\hline
	 					probablity  &  cost(ALG) \\
	 					\hline 
	 					$\frac{b-1}{b}$  & b\\  
						 & \\
			 			$\frac{1}{b}$     & k \\
			 			
			 	
				
			 	
			 		\hline 
			 

			 
				\end{tabular}
			\end{center}
			
			$$\mathop{\mathbb{E}}(ALG)=\frac{b-1}{b} \times b + \frac{1}{b} \times k $$
			$$ = b +\frac{k}{b}-1$$
		\end{quote}
		
		2. Does the Algorithm achieve constant competitive ratio?\\
		\begin{quote}
			There are 2 Cases for different values of OPT cost\\
				a) \textit{k}$\leq$ b ---> OPT=k\\ 
					\begin{quote}
						$$\frac{\mathop{\mathbb{E}}(ALG)}{OPT}=\frac{b+\frac{k}{b}-1}{k}$$
						$$=\lim_{k \to \infty } \frac{b}{k} + \frac{1}{b} - \frac{1}{k}$$
						
						$$= \frac{1}{b}$$

					\end{quote}
				b) \textit{k}>b  --->OPT=b\\
					\begin{quote}
						$$\frac{\mathop{\mathbb{E}}(ALG)}{OPT}=\frac{b+\frac{k}{b}-1}{b}$$
						$$=\lim_{k \to \infty } 1+ \frac{k}{b^2} - \frac{1}{b}$$
						
						$$=\infty$$
					\end{quote}
			Thus there is no constant competitive ratio.
		\end{quote}
		3. New Randomization Algorithm. What is the $\rho$?
		\begin{quote}
			Given:\\
			Rent equipment for b-1 days. For each of the following day, buy epuipment with a chance of 1/3 and continue renting otherwise.
			If you decide to buy on a particular day you don't make decisions the following day.\\
			
			let \textit{k} be the number of days of  AFTER b-1 days that the person can ski.$k\in W $
			
			P(Buy)=1/3  (After b-1 days)\\

			
		\begin{center}
			\begin{tabular}{ | c | c | c | c |}
				\hline
	 				days(d)  &  probablity  &  cost(ALG) & cost(OPT)\\
	 				\hline 
			 		$(b-1)$        & 1                                                            &  $(b-1)$                  &   $(b-1)$\\
			 		$(b-1)+1$   &  $\frac{1}{3}$                                       &  $(b-1) + b$           &   $b$ \\
			 		$(b-1)+2$   &  $\frac{1}{3}\times \frac{2}{3} $         & $(b-1)+(1+b)$      &   $b$ \\ 
			 		$(b-1)+3$   &  $\frac{1}{3}\times (\frac{2}{3})^2 $  &  $(b-1)+(2+b)$     &   $b$ \\
			 		$(b-1)+4$   &  $\frac{1}{3}\times (\frac{2}{3})^3 $  &  $(b-1)+(3+b)$     &   $b$ \\
					 .                 &  .                                                             &  .                           & .\\
					 .                 &  .                                                             &  .                           & .\\

			 		$(b-1)+k$   &  $\frac{1}{3}\times (\frac{2}{3})^{k-1} $  &  $(b-1)+(k+b)$     &   $b$ \\			 		
			 		

				
			 	
			 	\hline 
			 

			 
			\end{tabular}
		\end{center}
		

		$$\mathop{\mathbb{E}}(ALG)= \sum_{k=0}^{\infty}((b-1) + (k+b) )\times (\frac{2}{3})^{k-1} \times  \frac{1}{3}$$
		
		$$\mathop{\mathbb{E}}(ALG)= \sum_{k=0}^{\infty}(2b+k-1 )\times (\frac{2}{3})^{k} \times  \frac{1}{2}$$
		
		$$\mathop{\mathbb{E}}(ALG)=3b + \frac{1}{2}$$
		\end{quote}
	\end{quote}




%\end{subs}

%----------------------------------------------------------------------------------------------------------------------------------------


\newpage



%----------------------------------------------------------------------------------------------------------------------------------------
\section*{3.     Randomized Line Search Problem} 
\begin{subs}
	\begin{quote}
		P( initial direction to be +1) = $\frac{1}{2}$\\
		P(initial direction to be -1) = $\frac{1}{2}$\\
		Rest of the startegy remains the same.
		
		Suppose that the object has been placed at distance $d\geq 1$ from the origin.\\
		OPT=d.\\
		In phase i the robot visits location $(-2)^i$ and \\travels distance $(-2)^i \times 2$\\
		The worst case is when an object is located just outside of the radius covered in some phase.\\
		Then the robot returns to the origin, doubles the distance and travels in the “wrong direction”,
		 returns to the origin, and discovers the object by travelling in the “right direction.”\\
		In other words when the object is at distance $d, 2^i <d\leq2^{i+1} $ in the direction $(-1)^i$\\
		The total distance travelled is \\
		$$ 2(1+2+....+2^i+2^{i+1})+d \leq 2\times2^{i+2} +d<8d+d=9d$$
		This doubling strategy gives a 9-competitive algorithm for the line search problem.	
		\bigbreak

		From our analysis of the deterministic algorithm, we see that we got lucky when the robot started walking in
the "good" direction at the beginning of the algorithm. Since the adversary can see our algorithm, however,
he always chooses to put the door in the "bad" direction. Thus, a natural way to improve the competitive
ratio of our algorithm is to 
ip a fair coin to decide which direction to start walking in. If the adversary is
oblivious, then he doesn't know the outcome of the coin 
ip, and we now have an equal chance of getting
lucky or unlucky.
	\end{quote}

\begin{quote}
	

		
	     
\end{quote}




\end{subs}

%----------------------------------------------------------------------------------------------------------------------------------------



\newpage



\end{document}